% Options for packages loaded elsewhere
\PassOptionsToPackage{unicode}{hyperref}
\PassOptionsToPackage{hyphens}{url}
%
\documentclass[
]{book}
\usepackage{amsmath,amssymb}
\usepackage{iftex}
\ifPDFTeX
  \usepackage[T1]{fontenc}
  \usepackage[utf8]{inputenc}
  \usepackage{textcomp} % provide euro and other symbols
\else % if luatex or xetex
  \usepackage{unicode-math} % this also loads fontspec
  \defaultfontfeatures{Scale=MatchLowercase}
  \defaultfontfeatures[\rmfamily]{Ligatures=TeX,Scale=1}
\fi
\usepackage{lmodern}
\ifPDFTeX\else
  % xetex/luatex font selection
\fi
% Use upquote if available, for straight quotes in verbatim environments
\IfFileExists{upquote.sty}{\usepackage{upquote}}{}
\IfFileExists{microtype.sty}{% use microtype if available
  \usepackage[]{microtype}
  \UseMicrotypeSet[protrusion]{basicmath} % disable protrusion for tt fonts
}{}
\makeatletter
\@ifundefined{KOMAClassName}{% if non-KOMA class
  \IfFileExists{parskip.sty}{%
    \usepackage{parskip}
  }{% else
    \setlength{\parindent}{0pt}
    \setlength{\parskip}{6pt plus 2pt minus 1pt}}
}{% if KOMA class
  \KOMAoptions{parskip=half}}
\makeatother
\usepackage{xcolor}
\usepackage{longtable,booktabs,array}
\usepackage{calc} % for calculating minipage widths
% Correct order of tables after \paragraph or \subparagraph
\usepackage{etoolbox}
\makeatletter
\patchcmd\longtable{\par}{\if@noskipsec\mbox{}\fi\par}{}{}
\makeatother
% Allow footnotes in longtable head/foot
\IfFileExists{footnotehyper.sty}{\usepackage{footnotehyper}}{\usepackage{footnote}}
\makesavenoteenv{longtable}
\usepackage{graphicx}
\makeatletter
\def\maxwidth{\ifdim\Gin@nat@width>\linewidth\linewidth\else\Gin@nat@width\fi}
\def\maxheight{\ifdim\Gin@nat@height>\textheight\textheight\else\Gin@nat@height\fi}
\makeatother
% Scale images if necessary, so that they will not overflow the page
% margins by default, and it is still possible to overwrite the defaults
% using explicit options in \includegraphics[width, height, ...]{}
\setkeys{Gin}{width=\maxwidth,height=\maxheight,keepaspectratio}
% Set default figure placement to htbp
\makeatletter
\def\fps@figure{htbp}
\makeatother
\setlength{\emergencystretch}{3em} % prevent overfull lines
\providecommand{\tightlist}{%
  \setlength{\itemsep}{0pt}\setlength{\parskip}{0pt}}
\setcounter{secnumdepth}{5}
\usepackage{booktabs}
\usepackage{amsthm}
\makeatletter
\def\thm@space@setup{%
  \thm@preskip=8pt plus 2pt minus 4pt
  \thm@postskip=\thm@preskip
}
\makeatother
\ifLuaTeX
  \usepackage{selnolig}  % disable illegal ligatures
\fi
\usepackage[]{natbib}
\bibliographystyle{apalike}
\usepackage{bookmark}
\IfFileExists{xurl.sty}{\usepackage{xurl}}{} % add URL line breaks if available
\urlstyle{same}
\hypersetup{
  pdftitle={Computer Science Placement Handbook 2025/6},
  pdfauthor={A guidebook for students},
  hidelinks,
  pdfcreator={LaTeX via pandoc}}

\title{Computer Science Placement Handbook 2025/6}
\author{A guidebook for students}
\date{Last updated on 04 May, 2025}

\begin{document}
\maketitle

{
\setcounter{tocdepth}{1}
\tableofcontents
}
\chapter*{Welcome}\label{welcome}
\addcontentsline{toc}{chapter}{Welcome}

Welcome to the placement manual

\begin{itemize}
\tightlist
\item
  Duncan Hull (employability lead)
\item
  David Petrescu (Industrial Experience tutor)
\end{itemize}

\chapter{Introduction}\label{intro}

Studying engineering at the University of Manchester helps students to gain technical skills and knowlege in lectures, laboratories, and during projects both in individual and team-based roles. With this engineering knowledge students will be able to solve problems, develop new ideas, and design innovative solutions to solve a wide spectrum of engineering and social problems.

While working for an employer, graduate engineers gain valuable experience and in many cases discover what they really like and what to focus their working life on in the long term. The ``with Industrial Experience'' (wIE) scheme of our courses provides a valuable opportunity for students to obtain experience working as an early-career engineer in the real world within the period of their degree programme. There are many advantages to this, including:

\begin{itemize}
\tightlist
\item
  The experience of industrially focused engineering and applying it to real-world scenarios
\item
  The responsibilities associated with industrial employment.
\item
  Working within a team
\item
  The satisfaction of contributing to engineering products that will influence the future development of society.
\item
  The consolidation of a education with that of the engineering environment
\item
  The increased likelihood of job offers after graduation.
\item
  For many, the year in industry is the transformation from student to engineer, from student to professional
\end{itemize}

There are many advantages for employers who host students on placmeent

\begin{itemize}
\tightlist
\item
  The opportunity to have a year-long ``interviews'' with undergraduates who have two years (or more) experience at university.
\item
  The ability to familiarise students with in-house methods leading to fast-track interviewing and graduate
  training as a prospective future employee.
\item
  Access to high quality students as industrial trainees who can then offer the company valuable resources and new ideas.
\item
  Employers with a long-term commitment to the placement of students will have access to future potential recruits by maintaining contact with the Department through the wIE team.
\end{itemize}

We hope that you enjoy and make the most of your placement year in industry and wish you the best of luck!

\section{TODO: What is a placement}\label{todo-what-is-a-placement}

\chapter{Aims and Intended Learning Outcomes}\label{aims-and-intended-learning-outcomes}

The Learning outcomes for the placement year are

\begin{itemize}
\tightlist
\item
  To learn how to operate in a professional environment
\item
  To apply the skills and knowledge you've learned at University in the workplace
\item
  To grow as a professional by seeking out development opportunities to acquire new skills and knowledge in the workplace
\item
  To meet (or exceed) the expectation of your employer, as set out in your contract of employment
\item
  To describe your development as a professional in a short written report
\end{itemize}

Some of this will involve developing skills and knowledge beyond your University curriculum such as softer skills and digital capabilities, described below.

\section{Developing your digital capabilities}\label{developing-your-digital-capabilities}

Throughout your time at The University of Manchester you will be supported to develop the digital capabilities needed for your studies and future careers. Your placement experience will provide you with further authentic digital development opportunities. The following information is intended to help
you record and articulate your capabilities, supporting your development and further enhancing your employability.

\section{What are digital capabilities?}\label{what-are-digital-capabilities}

Digital capabilities enable us to `live, learn and work in a digital society.' This profile provides further information about the capabilities University students might expect to develop; it is important to contextualise these, so they are relevant to you and your studies.

\section{Exploring your digital capabilities}\label{exploring-your-digital-capabilities}

The Jisc Discovery tool is a supportive online tool, that can help you understand and develop your digital capabilities. By completing a question set within the tool you can gain a personalised report that includes
links to suggested resources in the tool's resource bank to support your further development. It is recommended that you repeat the question sets annually, to recognise your progress across the different elements of digital capability. You can access the Discovery tool through

\begin{itemize}
\tightlist
\item
  My Learning Essentials: Develop your digital capabilities resource. \href{https://www.education.library.manchester.ac.uk/none-programme-content/digital-capabilities/}{education.library.manchester.ac.uk/none-programme-content/digital-capabilities/}
\item
  My Skills Development -- on CareerConnect \href{https://www.careers.manchester.ac.uk/findjobs/skills/myskills/}{careers.manchester.ac.uk/findjobs/skills/myskills}
\end{itemize}

These resources support you in reflecting on your development on your reports - providing an action plan template for you to complete.

For placement students, we recommend taking the:

\begin{itemize}
\tightlist
\item
  `Current student' digital capability question set, which provides an in-depth exploration of your digital confidence and experience.
\item
  `Digital skills in AI and generative AI' question set, which also has its own resource bank
\end{itemize}

\section{Recording your development}\label{recording-your-development}

From work with employers, we know they value narratives around how you have developed your skills and capabilities through your studies. You can use the language from your Discovery tool reports to update your CV / online professional profile to help you record your digital development. This blog by the Careers Service can further support you with capturing and articulating the capabilities you develop during your placement. \citep{conway}

\chapter{Requirements for Industrial Experience}\label{requirements-for-industrial-experience}

The requirements for Industrial Experience are:

\begin{enumerate}
\def\labelenumi{\arabic{enumi}.}
\tightlist
\item
  You should be registered on IE, change degree \href{studentnet.cs.manchester.ac.uk/ugt/changedegree.php}{}
\item
  You will need to prepare and debug your CV, as covered in COMP101 and COMP2CARS, see \href{https://www.cdyf.me/debugging}{Debugging your Future} \citep{debugging}
\item
  You need to find and apply for jobs, see \href{https://www.cdyf.me/finding}{Finding your Future} \citep{finding}
\item
  You will need to have job interviews
\item
  You will need to accept an offer and then tell us about it, BEFORE you sign any contract of employment
\item
  Once accepted, do a skills audit
\end{enumerate}

There are no minimum grade requirements besides passing the first and second year of your degree. The placement year doesn't count towards your degree classification, but it \textbf{does} appear in the title of your
degree e.g.

\begin{itemize}
\tightlist
\item
  \texttt{BSc\ (Hons)\ Computer\ Science\ with\ Industrial\ Experience}: instead of
\item
  \texttt{BSc\ (Hons)\ Computer\ Science}: the ``vanilla`` degree
\end{itemize}

You MUST do the following if you want to do Industrial Experience (IE):

\begin{enumerate}
\def\labelenumi{\arabic{enumi}.}
\tightlist
\item
  \textbf{CHECK} your visa. If you're not a UK or EU student and have a Student visa, you will need to extend your visa an extra year to include your placement year if you're not already registered on IE. See \href{documents.manchester.ac.uk/display.aspx?DocID=37044}{Student Visas: Changing your Course} \citep{changing}
\item
  \textbf{REGISTER} for the appropriate degree programme with IE (if you're not already) by completing a programme change form. You can do this at any point during your second year Bachelors (or third year MEng) up until the end of July of the year that you start your placement
\item
  \textbf{FIND} an appropriate job and go to interviews throughout your second year of your Bachelors degree (or third year of MEng) see above. Your placement needs to be something broadly related to your degree but this can encompass a wide variety of activities from business analysis, software engineering, software testing, hardware design, data science etc. You can self-arrange a placement but it needs to be approved before you can go on placement
\item
  \textbf{SUBMIT} your placement application as soon as you've received an offer of employment by filling in a form at MyPlacement: \href{studentmobility.manchester.ac.uk}{https://studentmobility.manchester.ac.uk}. You can do this by clicking on ``work experiences'' \textgreater{} ``computer science''. The Placement Journey infographic takes you through the process step by step. Placements outside the UK are subject to additional checks and approval, for example we need to ensure employers have employers liability insurance or an equivalent
\item
  \textbf{WAIT} for our approval BEFORE you make any concrete arrangements to start work, such as signing a contract of employment, moving house, flying to another country etc
\end{enumerate}

\chapter{Your responsibilities}\label{you}

As a student, you are expected to complete documents for the University as part of the myPlacement application at \href{https://studentmobility.manchester.ac.uk/}{studentmobility.manchester.ac.uk/}. This includes: UNIV+: Work Placement Declaration (subject to updates, check online for latest version). This document sets out some of your responsibilities. When you sign the document, you agree to the following:

\begin{enumerate}
\def\labelenumi{\arabic{enumi}.}
\tightlist
\item
  You will declare any disability or serious, unstable or difficult to manage physical or mental health conditions to my placement administrator at the earliest opportunity. You understand that my placement administrator will work alongside specialist services, such as the Disability Advisory and Support Service and Occupational Health, to investigate the support systems available to you from your employer and any additional funding that you may be entitled to.
\item
  For your attention: By sharing information with The University of Manchester about a disability or serious, unstable or difficult to manage physical or mental health conditions, you enable us to provide any necessary additional support during the application process, pre-departure preparation and during your placement.
\item
  It is rare that a health condition or disability would result in you being unable to participate in a work placement, but you should be aware that the ability to support health conditions or disabilities varies significantly by country.
\item
  If you have an approved mitigating circumstances case, you will inform your placement administrator at the earliest opportunity.
\item
  If your placement application is approved, you understand that you will need to complete all requirements, such as the health needs self-assessment. You will take preparation for your placement seriously and understand that it
  will require a considerable time commitment.
\item
  You declare that the information presented in your my placement application and the accompanying documentation is true and complete.
\end{enumerate}

  \bibliography{book.bib}

\end{document}
